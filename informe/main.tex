\documentclass[onecolumn, amsmath, amssymb, aps]{revtex4-2}

%\usepackage{silence}
%\usepackage{preamble}

\usepackage{physics}
\usepackage{float}
\usepackage{xcolor}
\usepackage{enumerate}
\usepackage{siunitx}
\usepackage[makeroom]{cancel}
\usepackage[framemethod=tikz]{mdframed}
\usepackage{lmodern}
\usepackage{upgreek}
\usepackage{dsfont}
\usepackage{wrapfig} % wrap figure in text
\usepackage{geometry}
%\usepackage{caption}
\usepackage[pdf]{graphviz}
\usepackage[most]{tcolorbox}
% Don't indent paragraphs, leave some space between them
%\usepackage{parskip}
\usepackage[]{quoting}
\usepackage{subfiles}
\usepackage[caption=false]{subfig}
\usepackage{dcolumn}% Align table columns on decimal point

\raggedbottom

\definecolor{alert-color}{rgb}{0.82, 0.1, 0.26} % alizarin
\newcommand{\Cr}[1]{\textbf{\textcolor{alert-color}{#1}}}


\def\bibsection{\section*{Referencias}} % delete ugly reference line

\renewcommand{\thesection}{\arabic{section}}
\renewcommand{\thesubsection}{\thesection.\arabic{subsection}}
\renewcommand{\thesubsubsection}{\thesubsection.\arabic{subsubsection}}
\renewcommand\tablename{TABLA}

\begin{document}
\subfile{sections/caratula_y_firmas.tex}

\title{Laboratorio 6 y 7 - Modulación de fase en Ondas Faraday y oscilones}
\author{Bernardo Español}
\email{esp.bernardo@gmail.com}

\author{Melisa Vinograd}
\email{melivinograd@gmail.com}

\affiliation{
Universidad de Buenos Aires}

\author{Dr. Pablo Cobelli}
\email{cobelli@df.uba.ar}
\affiliation{
Laboratorio de Turbulencia Geofísica, FLiP: Fluidos y Plasmas}

\section*{Resumen}
\subfile{sections/resumen.tex}

\section{Introducción}
\subfile{sections/introduccion.tex}

\section{Montaje experimental}
\subfile{sections/montaje_experimental.tex}

\section{Extracción de la superficie libre}
\subfile{sections/extraccion_de_la_superficie_libre.tex}

\section{Resultados}
\subfile{sections/resultados.tex}

\section{Sumario y perspectivas}
\subfile{sections/sumario_y_perspectivas.tex}

\section*{Anexo}
\subfile{sections/anexo.tex}

% Referencias
\bibliographystyle{unsrt}
\bibliography{bibliography}


%\section*{Punteo del informe}
%\begin{itemize}
%\item Un largo total de 15 carillas de contenido parece razonable.
%\item Cuantificar la velocidad del algoritmo y cómo escala.
%\item Realizar un diagrama de flujo del algoritmo.
%\item Empezar por resultados.
%\end{itemize}

\end{document}
