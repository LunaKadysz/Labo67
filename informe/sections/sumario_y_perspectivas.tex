\documentclass[../main.tex]{subfiles}

\begin{document}
En este trabajo de Laboratorio 6 y 7 se logró poner en funcionamiento un montaje experimental que permite medir y observar el fenómeno de ondas de Faraday en un sistema unidimensional con y sin modulación de fase. A su vez se desarrollaron algoritmos para procesamiento de imágenes y un protocolo de medición que permiten obtener la altura y dinámica de la superficie libre. Las mejoras y optimizaciones del algoritmo, así como la paralelización del mismo hacen que sea factible pensar en utilizar alguno de los dos clusters Beowulf que posee el Grupo de Fluidos y Plasmas para aprovechar la escalabilidad en términos de tiempo de cómputo. A su vez, se pudieron observar oscilones suaves, propagativos como no propagativos, fuertes (estrictamente propagativos) y modulación de fase para algunos regímenes del forzado.

Los resultados muestran que el montaje experimental y las técnicas desarrolladas permiten el estudio cuantitativo del fenómeno en profundidad. Se realizaron mediciones que cubren un amplio rango de parámetros del sistema. Tenemos un sistema muy rico, que tienen comportamientos fuertemente no lineales para distintos modos de excitación. Pudimos comparar, reproducir y estudiar cada uno.  Para la descripción de la dinámica se pudo  separar el problema en dos escalas de tiempo: una asociada a la oscilación rápida de las ondas superficiales y otra vinculada con la modulación de las amplitudes locales de dichas ondas. 

Pudimos estudiar estructuras desde el orden de los oscilones individuales, caracterizando su geometría y su velocidad, hasta fenómenos de grupo de longitudes de onda comparables con el anillo. En relación a esto último, se trabajó con las envolventes espacio temporales del fenómeno estudiando escalas temporales más largas. Se comprobó que a partir de este enfoque surgen inestabilidades conocidas en otros sistemas no lineales como es Taylor-Couette. La formación de este patrón global se pudo explicar mediante la contribución de ondas propagativas a izquierda y derecha


Por último, se desarrollaron herramientas que permiten estudiar la distribución de energía y su posterior comparación con los modelos teóricos, hallando una discrepancia con la relación de dispersión de ondas gravito-capilares para aguas poco profundas. Ésta se logró explicar cualitativamente la no-linealidad del problema.  

Dada la robustez del método de medición y análisis, se tienen varias perspectivas a corto plazo. Resulta de particular interés la potencialidad que el sistema de medición empleado (FTP) representa para la determinación experimental de los coeficientes de las ecuaciones de amplitud. Esto se sustenta en que la fuerte evidencia de que estos modelos funcionan para sistemas que forman estructuras no locales propagativas. 

De forma general sería interesante poder predecir la velocidad o amplitud de los oscilones en función de la aceleración de excitación. Es decir, hacer un paneo general en alto margen de amplitudes para hallar relación entre ambas magnitudes. 

De cara a estudiar la inestabilidad de Taylor-Couette en profundidad se podría realizar un barrido de los parámetros de excitación del sistema de modo de establecer la franja de aparición de la misma. 

% Con el objetivo de ajustar estos coeficientes se trabajará con Syndy (Sparse Identification of Nonlinear Dynamics). Se prestará especial atención a identificar la dinámica general del problema: analizar, por ejemplo, coeficientes que acompañan a las segundas derivadas, tanto espaciales como temporales, para ver el efecto de la disipación en el sistema. Así mismo se buscará predecir la longitud de onda característica y velocidades de propagación de los oscilones y estimar parámetros del sistema para los cuales se observan estos fenómenos.
% \begin{itemize}
% 	\item Medir con esta técnica para luego modelar es factible.
% 	\item Escribir lo del plan. En particular buscamos lanzar una campaña de mediciones en un buen cacho del espacio de parámetros. Modelar la envolvente.
% 	\item Quizás hablar sobre modelar la superficie libro con un GL o un SNL.
% 	\item No solo buscamos fittear los coeficientes, sino el modelo. (La identificación de la dinámica del problema).
% \end{itemize}



\end{document}
