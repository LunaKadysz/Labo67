\documentclass[../main.tex]{subfiles}

\begin{document}
En el marco del estudio de ondas de Faraday y oscilones este trabajo presenta un montaje experimental que permite estudiar este fenómeno para el caso unidimensional. Para lo mismo trabajamos en una celda de tipo anular con condiciones de contorno periódicas forzada verticalmente. Implementamos y mejoramos una técnica de profilometría de la superficie libre, así como algoritmos para el procesamiento de los datos. A lo largo de este informe se muestran las distintas dinámicas observadas del sistema para varias configuraciones de excitación.
Se estudian la energía y el comportamiento de las estructuras emergentes a distintas escalas espacio-temporales. 
Estos resultados muestran que el montaje experimental y las técnicas desarrolladas hacen factible el estudio del fenómeno en profundidad.
\end{document}